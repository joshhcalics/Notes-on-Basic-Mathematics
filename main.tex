\documentclass{amsbook}

\begin{document}

\title{Notes on Basic Mathematics}
\author{Josh H}
\maketitle

\chapter{The integers}
\section{Rules for addition}
	We assume proficiency with the addition operator.
	There exists a 

\chapter{Properties}
\section{Multiplication; Associative}
	The associative property.

	\begin{align*}
		a(bc) 	&= (ab)c		\\
				&= abc			\\
				&= (a)(b)(c)	\\
				&\hspace{5pt} \vdots
	\end{align*}

	Assume \textit{d} can be expanded to \textit{bc}.

	\begin{align*}
		a(d)&= ad			\\
			&= a(bc) = abc
	\end{align*}

	Expressions within parentheses or have binding with an exponent are to be
	evaluated before the distributions are applied.

	\begin{align*}
		a(b + c)^2 	&= 		a(b^2 + 2bc + c^2)			\\
					&\ne	(ab + ac)^2					\\
					&\ne	a(b^2 + c^2) = ab^2 + ac^2	
	\end{align*}

	Expansions mixed with distributions can be syntactically difficult.

	\begin{flalign*}
		a(b^2 - c^2)&= 		a(b - c)(b + c)		& \text{is equivalent to,}			\\
					&= 		a((b - c)(b + c))	& \text{but not to be confused with,}	\\
					&\ne	a(b - c) * a(b + c)	\\
	\end{flalign*}
	

\chapter{Inequalities}
	The LHS has a relation to the RHS; it is either greater than, less than,
	greater than or equal, or less than or equal to.

	\begin{align*}
		5	&< 		10	\\
		10	&>		5	\\
		-13	&\leq	3	\\
		3	&\geq	3
	\end{align*}

	This is an example of an inequality.

	\begin{align*}
		x + 1 	&< 5		\\
		x		&< 5 - 1	\\
		x		&< 4
	\end{align*}

	Another.

	\begin{flalign*}
		x^2 - 6x &\geq 0	& \text{factor expression,}	\\
		x(x - 6) &\geq 0	& \text{observe when zero,}	\\
		x = 0, \quad &x = 6
	\end{flalign*}

	We have found the endpoints on the number line.
	Now we select a number between the intervals separated by the endpoints.
	These numbers are picked arbitrarily.
	Here, the chosen numbers \textit{z} must follow the rule.

	\begin{align*}
		\text{endpoints}&
		\begin{cases}
			0 \\ 6
		\end{cases}	\\
		\text{intervals}&
		\begin{cases}
			-1& \text{$-\infty < z < 0$}	\\
			1&	\text{$0 < z < 6$}			\\
			7&	\text{$6 < z < \infty$}
		\end{cases}
	\end{align*}

	We then substitute those numbers in each factor to determine the sign
	of the expression.
	If the sign of the LHS is correct in relation to the type of inequality
	then that interval is valid.
	Since this inequality also determines if the LHS is equal to 0 then
	we check the endpoints as well.

	\begin{equation*}
		(-\infty, 0] \cup [6, \infty)
	\end{equation*}
\end{document}