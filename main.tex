\documentclass{amsbook}

\begin{document}
	\begin{titlepage}
		\centering
		\Huge{\normalfont{\textsc{Notes on Basic Mathematics}}} \par \vspace{8cm}
		\small{\textit{written by}} \par
		\Large{Josh H}
	\end{titlepage}

\chapter{The integers}
\section{Rules for addition}
	We assume proficiency with the addition operator.
	Observe the rule for addition with $0$, that is
	\begin{equation} \label{eq:adding_zero}
		a + 0 = a.
	\end{equation}

	The formula for the addition of negative numbers is as follows,
	\begin{equation} \label{eq:adding_negatives}
		a + (-a) = 0 \quad\quad \text{and also} \quad\quad -a + a = 0.
	\end{equation}

	\begin{equation}
		\text{If} \quad a + b = 0, \quad \text{then} \quad b = -a \quad \text{and} \quad
		a = -b.
	\end{equation}
	
	\begin{equation}
		a = -(-a).
	\end{equation}
	
\chapter{Properties}
\section{Multiplication; Associative}
	The associative property.

	\begin{align*}
		a(bc) 	&= (ab)c		\\
				&= abc			\\
				&= (a)(b)(c)	\\
				&\hspace{5pt} \vdots
	\end{align*}

	Assume \textit{d} can be expanded to \textit{bc}.

	\begin{align*}
		a(d)&= ad			\\
			&= a(bc) = abc
	\end{align*}

	Expressions within parentheses or have binding with an exponent are to be
	evaluated before the distributions are applied.

	\begin{align*}
		a(b + c)^2 	&= 		a(b^2 + 2bc + c^2)			\\
					&\ne	(ab + ac)^2					\\
					&\ne	a(b^2 + c^2) = ab^2 + ac^2	
	\end{align*}

	Expansions mixed with distributions can be syntactically difficult.

	\begin{flalign*}
		a(b^2 - c^2)&= 		a(b - c)(b + c)		& \text{is equivalent to,}			\\
					&= 		a((b - c)(b + c))	& \text{but not to be confused with,}	\\
					&\ne	a(b - c) * a(b + c)	\\
	\end{flalign*}
	

\chapter{Inequalities}
	The LHS has a relation to the RHS; it is either greater than, less than,
	greater than or equal, or less than or equal to.

	\begin{align*}
		5	&< 		10	\\
		10	&>		5	\\
		-13	&\leq	3	\\
		3	&\geq	3
	\end{align*}

	This is an example of an inequality.

	\begin{align*}
		x + 1 	&< 5		\\
		x		&< 5 - 1	\\
		x		&< 4
	\end{align*}

	Another.

	\begin{flalign*}
		x^2 - 6x &\geq 0	& \text{factor expression,}	\\
		x(x - 6) &\geq 0	& \text{observe when zero,}	\\
		x = 0, \quad &x = 6
	\end{flalign*}

	We have found the endpoints on the number line.
	Now we select a number between the intervals separated by the endpoints.
	These numbers are picked arbitrarily.
	Here, the chosen numbers \textit{z} must follow the rule.

	\begin{align*}
		\text{endpoints}&
		\begin{cases}
			0 \\ 6
		\end{cases}	\\
		\text{intervals}&
		\begin{cases}
			-1& \text{$-\infty < z < 0$}	\\
			1&	\text{$0 < z < 6$}			\\
			7&	\text{$6 < z < \infty$}
		\end{cases}
	\end{align*}

	We then substitute those numbers in each factor to determine the sign
	of the expression.
	If the sign of the LHS is correct in relation to the type of inequality
	then that interval is valid.
	Since this inequality also determines if the LHS is equal to 0 then
	we check the endpoints as well.

	\begin{equation*}
		(-\infty, 0] \cup [6, \infty)
	\end{equation*}

\chapter{Polynomials}
\section{The constant function}
\section{The linear function}
\section{The quadratic function}

\section{The polynomial function}
\subsection{Characteristics}
	\begin{itemize}
		\item $a_n, a_{n-1}, a_{n-2}, \ldots, a_0$ are the coefficients.
		\item $a_0$ is the constant coefficient or constant term.
		\item $a_n$ is the leading coefficient.
		\item $a_nx^n$ is the leading term.
	\end{itemize}

	The behavior of polynomials is based on the leading term's sign and degree.

\subsection{End behavior}
	Graphs of polynomial functions are continuous.
	That is, the graphs have no breaks or holes.
	Additionally, the graphs have no cusps or corners; the graphs are smooth.

	Let P be a polynomial of degree $n$.
	\begin{center}
	\begin{tabular}{llll}
		\multicolumn{2}{c}{$n$ is odd} & \multicolumn{2}{c}{$n$ is even} \\ \hline
		$a_n$ is positive & $a_n$ is negative & $a_n$ is positive & $a_n$ is negative \\ \hline

		$P(x) \rightarrow \infty$ 	& $P(x) \rightarrow -\infty$
		& $P(x) \rightarrow \infty$	& $P(x) \rightarrow -\infty$ \\
		$x \rightarrow \infty$		& $x \rightarrow \infty$
		& $x \rightarrow \infty$	& $x \rightarrow \infty$ \\ \hline

		$P(x) \rightarrow -\infty$	& $P(x) \rightarrow \infty$
		& $P(x) \rightarrow \infty$	& $P(x) \rightarrow -\infty$ \\
		$x \rightarrow -\infty$		& $x \rightarrow -\infty$
		& $x \rightarrow -\infty$	& $x \rightarrow -\infty$
	\end{tabular}
	\end{center}

\subsection{Multiplicity}
	The multiplicity of a zero $c$ determines the shape of the graph of $P$ near $c$.
	\begin{equation*}
		P(x) = x^3(x-2)^2
	\end{equation*}
	Here, $c = 0$ or $c = 2$.
	The multiplicities of $0$ and $2$ are $3$ and $2$, respectively.
	All we are doing is looking at the exponents.
	When the multiplicity of $c$ is even the graph touches the x axis.
	When the multiplicity of $c$ is odd the graph crosses the x axis.
\end{document}